\setlength{\parskip}{\baselineskip} 
\section{Conclusion}

% mixtures in general
% pcp
% bnmf
% edcs and iq